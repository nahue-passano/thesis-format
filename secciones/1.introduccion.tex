\subsection{FUNDAMENTACIÓN}

Breve fundamentación del trabajo de tesis en forma clara y concisa, estableciendo la relevancia y pertinencia del mismo en el ámbito del conocimiento donde se desarrolle. Debería establecerse la ubicación del trabajo dentro del estado del arte, pero sin desarrollar el detalle que se hará en los siguientes capítulos. No deben incluirse resultados ni detalles experimentales. Es deseable que no supere las 4 hojas.

\subsection{OBJETIVOS}

\subsubsection{Objetivo general}

Resumir en una sola frase el objetivo de la investigación.

\subsubsection{Objetivo especifico}

\begin{itemize}
    \item Listar los distintos objetivos siguiendo un orden metodológico. Recordar que la revisión bibliográfica no es un objetivo específico
    \item Identificar los recintos, paramentos verticales y horizontales y revestimientos fonoabsorbentes que conforman el estudio de grabación.
    \item Realizar simulaciones de tiempo de reverberación e indicadores de inteligibilidad de la palabra mediante el programa EASE.
    \item Obtener el índice de reducción sonora R de los distintos detalles constructivos mediante el programa de predicción INSUL.
    \item 	Realizar mediciones in situ de los descriptores de acondicionamiento acústico según norma ISO 3382-1.
    \item	Realizar mediciones in situ según norma ISO 16283-1 para obtener los índices de aislamiento acústico a ruido aéreo.
    \item Comparar los resultados de las simulaciones respecto a los valores obtenidos de las mediciones.
    \item Realizar propuestas de mejora a implementar.
\end{itemize}

\subsection{ESTRUCTURA DE LA INVESTIGACIÓN}

Indicar en distintos párrafos el contenido de cada uno de los capítulos que conformarán la tesis.

En el capítulo 2 se presenta el marco teórico donde se detallan los parámetros e indicadores necesarios para el desarrollo de la investigación.


En el capítulo 3 se hace referencia a aquellas investigaciones vinculadas con el tema a desarrollar en esta tesis.


En el capítulo 4…
