El diseño de investigación implica desarrollar las previsiones generales del plan para el experimento, definiendo el tipo de investigación a realizar, su alcance, etc. Dado que en este ejercicio se trata de evaluar subjetivamente alguna variable, debe explicarse en términos amplios el plan a desarrollar, el que se pormenorizará en los puntos siguientes. Es importante explicar todo el diseño para que se tenga una visión completa, sin redundar en las explicaciones que van a realizarse en los puntos que siguen. De este modo, el lector o evaluador del proyecto, tienen una primera impresión de la totalidad.

\subsection{DISEÑO PRUEBA OBJETIVA: DEFINICIÓN DE VARIABLES. MUESTRA}

Está claro que, si la investigación no tiene una parte subjetiva, el título de este ítem será: \textbf{Diseño de la Investigación, variables y muestra.}

En esta parte deben definirse las variables que serán medidas, y cuáles son las que se estima llevar a la evaluación subjetiva. Debe incluirse un diagrama de los equipos a emplear en las mediciones y un diseño preciso de cómo se llevarán a cabo.

\subsection{DISEÑO PRUEBA SUBJETIVA: ENCUESTA Y MUESTRA (SI EXISTIERA)}

En primer lugar debe definirse si se recurrirá a una muestra probabilística o no probabilística, la cantidad de sujetos que se encuestarán, y la justificación de ese decisión. Aquí debe desarrollarse también de modo completo el cuestionario y el desarrollo de la encuesta, es decir definir las preguntas, las muestras que se presentarán a los encuestados, las series de pruebas, duración, etc. Debe incluso definirse con exactitud el recorrido que hará cada encuestado, el lugar, las características del equipamiento a emplear, etc.
