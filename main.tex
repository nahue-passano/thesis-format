\documentclass[12pt]{article}
\usepackage[a4paper]{geometry}
\usepackage[spanish,es-tabla]{babel}
\usepackage[utf8]{inputenc}
\usepackage[T1]{fontenc}
% \usepackage{fontspec}
\usepackage{graphicx}
\usepackage{times}
\usepackage{amsmath}
\usepackage{setspace}
\usepackage{enumitem}
\usepackage{fancyhdr}
\usepackage{lipsum}
\usepackage{parskip}
\usepackage[backend=biber, style=apa]{biblatex}
\addbibresource{bibliografia.bib}
\usepackage{caption}
\usepackage{floatrow}
\usepackage{titlesec}
\usepackage{titletoc}
\usepackage{appendix}
\usepackage{tabularray}
\usepackage{tocloft} 
\usepackage{titlesec}
\usepackage{physics}
\usepackage{changepage}
\usepackage[hidelinks]{hyperref} % eliminar rectangulos rojos en los links

\usepackage{titlesec}
\titleformat{\section}
  {\normalfont\fontsize{16}{16}\bfseries}{\thesection .}{0.5ex}{}
\titleformat{\subsection}
  {\normalfont\fontsize{14}{14}\bfseries}{\thesubsection .}{0.5ex}{}
\titleformat{\subsubsection}
  {\normalfont\fontsize{14}{14}\bfseries}{\thesubsubsection.}{0.5ex}{}


%-----------------CONFIGURACION-------------------------------

% Configuracion de la Pagina 
\geometry{
    a4paper,
    total={21cm,29.7cm},
    left=3cm,
    top=2.5cm,
    right=2.5cm,
    bottom=2.5cm }

\setstretch{1.5} %Interlineado a 1.5
\setlength{\parindent}{1cm} %Sangría de cada nuevo párrafo a 1cm

% Configuración del pie de página
\fancyfoot[C]{\rule{\textwidth}{0.4pt}\\ \hspace*{\fill}\fontsize{12pt}{14pt}\selectfont Título de Tesis \hspace*{\fill}\thepage} 

%\setlength{\footskip}{1.02 cm}
\setlength{\footskip}{0.5 cm}
\pagestyle{fancy} % Establecer el estilo de página como 'fancy'

\fancyhead{} % Limpiar encabezado
\renewcommand{\headrulewidth}{0pt} % Eliminar la línea de encabezado

\DefineBibliographyStrings{spanish}{ % Redefinir la cadena de texto para Bibliografía
  references = {BIBLIOGRAFÍA}, % <-- Aquí cerramos correctamente la definición}

% Configuración global de las descripciones de las figuras
\captionsetup[figure]{labelsep=period}
% Configuración global de las descripciones de las tablas
\floatsetup[table]{capposition=top}
\captionsetup[table]{labelsep=period}

% Personalización del formato del índice de figuras
\renewcommand{\cftfigpresnum}{Figura }
\renewcommand{\cftfigaftersnum}{.}
\setlength{\cftfignumwidth}{4em}

% Personalización del formato del índice de tablas
\renewcommand{\cfttabpresnum}{Tabla }
\renewcommand{\cfttabaftersnum}{.}
\setlength{\cfttabnumwidth}{4em}

% Personalizar el formato del índice general
\renewcommand{\cftsecleader}{\cftdotfill{\cftdotsep}} % Agrega puntos entre el título de la sección y el número de página
\renewcommand{\cftsecfont}{\bfseries} % Establece la fuente del título de la sección en el índice como negrita
\renewcommand{\cftsecpagefont}{\bfseries} % Establece la fuente de la página de la sección en el índice como negrita


% Definir el espaciado entre secciones y subsecciones
\titlespacing*{\subsection}{1cm}{*1.5}{*1}

% Definir el espaciado entre subsecciones y subsubsecciones
\titlespacing*{\subsubsection}{1.5cm}{*1.5}{*1}

\usepackage{fontspec}
\setromanfont[
BoldFont=Calibri Bold.TTF,
ItalicFont=Calibri Italic.ttf,
BoldItalicFont=Calibri Bold Italic.ttf,
]{Calibri Regular.ttf}

\begin{document}

%-------------------------- PORTADA -----------------
\begin{titlepage}
\centering
\includegraphics[width=13.58cm, height=3.1cm]{Logo Untref.png} 

\vspace{0.1cm}

\hspace*{-1.31cm}% Espacio horizontal de 1.31cm desde el borde izquierdo de la hoja
\begin{minipage}[t]{16cm}
\centering
\framebox[17.13cm][c]{\parbox{18.13cm}{\centering{\fontsize{16pt}{1.5pt}\selectfont INGENIERÍA EN SONIDO}}}
\vspace{0.5cm} % Espacio vertical entre el recuadro y el texto siguiente

\end{minipage}


\vspace{36pt}

{\bfseries\fontsize{22pt}{24pt} \selectfont Título de Tesis \par}

\vspace{22pt}

{\itshape\fontsize{18pt}{24pt}\selectfont \textbf{Subtítulo de la tesis (si lo tuviera)} \par}

\vspace{44pt}

{\centering\itshape\fontsize{14pt}{1pt}\selectfont Tesis final presentada para obtener el título de Ingeniero\par}
{\centering\itshape\fontsize{14pt}{1pt}\selectfont de Sonido de la Universidad Nacional de Tres de Febrero \par}
{\centering\itshape\fontsize{14pt}{1pt}\selectfont (UNTREF) \par}

\vspace{70pt}

{\bfseries\fontsize{14pt}{0pt}\selectfont TESISTA: Nombre y apellido (DNI número) \par}
{\bfseries\fontsize{14pt}{0pt}\selectfont TUTOR/A: Nombre y apellido (Ing., PhD., etc.) \par}
{\bfseries\fontsize{14pt}{0pt}\selectfont COTUTOR/A: Nombre y apellido (si lo tuviera) \par}

\vfill

\begin{table}[h]
\hrulefill \\ 
\begin{tabular}{c}
Fecha de defensa: mes y año $\lvert$ Locación (Ej. Saenz Peña), Argentina \\
\end{tabular}
\end{table}
\hrule
\end{titlepage}

\newpage
\thispagestyle{empty} % Página en blanco sin número de página, encabezado ni pie de página
\mbox{} % Contenido de la página en blanco
\newpage

%---------------------------------------------------------------
% CUERPO DEL DOCUMENTO
% Centrar título de sección

\pagenumbering{roman} % Comienza la numeración de página en números romanos
\setcounter{page}{2} % Establece el número de página según sea necesario

%-----------------Agradecimientos-----------------
\newpage 
\thispagestyle{plain} 
% \section*{AGRADECIMIENTOS} % Título centrado
\input{secciones/0.1.agradecimientos}

\clearpage

\newpage
\thispagestyle{plain} % Página en blanco sin número de página, encabezado ni pie de página
\mbox{} % Contenido de la página en blanco
\newpage

%-----------------Estado del Arte-----------------
\newpage
\thispagestyle{plain} 
% \section*{DEDICATORIA} 
\input{secciones/0.2.dedicatoria}


%------------------Indices------------------------
\newpage
\renewcommand{\contentsname}{ÍNDICE DE CONTENIDOS}
\tableofcontents   % Índice General
\listoffigures  % Índice de Figuras
\listoftables   % Índice de Tablas 


%------------------Resumen------------------------

\newpage
\thispagestyle{plain} 
% \section*{\centering RESUMEN} 
\addcontentsline{toc}{section}{RESUMEN} % Agregar RESUMEN al índice
\input{secciones/0.3.resumen}
% \end{titlepage}

%------------------Abstrac------------------------

\newpage
\thispagestyle{plain} 
% \section*{\centering ABSTRACT} 
\addcontentsline{toc}{section}{ABSTRACT} % Agregar ABSTRACT al índice
\input{secciones/0.4.abstrac}



%-----------------Introducción-----------------
\clearpage % Asegúrate de que la sección de introducción comience en una nueva página
\pagenumbering{arabic} % Restaurar numeración arábiga
\section{INTRODUCCIÓN} 
\input{secciones/1.introduccion}


\clearpage



%-----------------Marco Teorico-----------------
\newpage
\section{MARCO TEÓRICO} 
    En el Marco Teórico debemos incorporar la bibliografía, artículos de revistas, ponencias de congresos, links de Internet o todo aquello que haya contribuido a formar el cuerpo del saber sobre el que va a basarse la investigación, incorporando los procesos y ecuaciones necesarios.

Puede ser uno varios capítulos donde se detallen los parámetros, indicadores y conceptos teóricos referentes al tema a tratar. Se recomienda no utilizar conceptos muy básicos, como definición de nivel de presión sonora, ponderación A, etc.


%-----------------Estado del Arte-----------------
\newpage
\section{ESTADO DEL ARTE} 
    Puede ser uno o varios capítulos que desarrollen el estado del arte del área de conocimiento donde se inserta la tesis. La profundidad del enfoque en el tratamiento de los temas debe ser adecuado para el entendimiento posterior de los resultados y discusiones de la tesis. No es necesario que sea autocontenido, es recomendable el uso amplio de referencias a trabajos previos que se encuentren en la literatura abierta sobre el tema.

%-----------------Diseño de la investigacion-----------------
\newpage
\section{DISEÑO DE LA INVESTIGACIÓN} 
    El diseño de investigación implica desarrollar las previsiones generales del plan para el experimento, definiendo el tipo de investigación a realizar, su alcance, etc. Dado que en este ejercicio se trata de evaluar subjetivamente alguna variable, debe explicarse en términos amplios el plan a desarrollar, el que se pormenorizará en los puntos siguientes. Es importante explicar todo el diseño para que se tenga una visión completa, sin redundar en las explicaciones que van a realizarse en los puntos que siguen. De este modo, el lector o evaluador del proyecto, tienen una primera impresión de la totalidad.

\subsection{DISEÑO PRUEBA OBJETIVA: DEFINICIÓN DE VARIABLES. MUESTRA}

Está claro que, si la investigación no tiene una parte subjetiva, el título de este ítem será: \textbf{Diseño de la Investigación, variables y muestra.}

En esta parte deben definirse las variables que serán medidas, y cuáles son las que se estima llevar a la evaluación subjetiva. Debe incluirse un diagrama de los equipos a emplear en las mediciones y un diseño preciso de cómo se llevarán a cabo.

\subsection{DISEÑO PRUEBA SUBJETIVA: ENCUESTA Y MUESTRA (SI EXISTIERA)}

En primer lugar debe definirse si se recurrirá a una muestra probabilística o no probabilística, la cantidad de sujetos que se encuestarán, y la justificación de ese decisión. Aquí debe desarrollarse también de modo completo el cuestionario y el desarrollo de la encuesta, es decir definir las preguntas, las muestras que se presentarán a los encuestados, las series de pruebas, duración, etc. Debe incluso definirse con exactitud el recorrido que hará cada encuestado, el lugar, las características del equipamiento a emplear, etc.


%-----------------Validacion de las pruebas-----------------
\newpage
\section{VALIDACIÓN DE LAS PRUEBAS} 
    Las pruebas se validarán siguiendo los criterios estadísticos que permiten determinar la existencia o no de errores en alguna o algunas encuestas y considerar si son incluidas o no en el conjunto de respuestas. Como las pruebas no se realizarán todavía, deben plantearse los métodos elegidos y explicarlos.

%-----------------Analisis de los resultados-----------------
\newpage
\section{ANÁLISIS DE LOS RESULTADOS: APLICACIONES ESTADÍSTICAS} 
    \input{secciones/6.analisis_de_los_resultados}

%-----------------Conclusiones-----------------
\newpage
\section{CONCLUSIONES} 
    \input{secciones/7.conclusiones}

%-----------------Linea futuras de Invetigacion-----------------
\newpage
\section{LÍNEAS FUTURAS DE INVESTIGACIÓN} 
    Las líneas futuras de investigación dentro del plan, suponen expresar qué posibles cuestiones que se encontrarán se dejarán para ser profundizadas más adelante por otras investigaciones.
\textcolor{white}{
\parencite{Beranek2005ConcertHA} 
\parencite{busch2005noise}
\parencite{Call2007SoundP}
\parencite{Gardner2002}
\parencite{iso1996-2}
\parencite{kracht2007noise}
\parencite{Lawson2010SoundIntensity}
\parencite{MacLeod2007QuietingWeinberg}
\parencite{Mazer2012Creating}
\parencite{Orellana2006NoiseIT}
\parencite{richardson2009development}
\parencite{Ryherd2008CharacterizingNA}
\parencite{sanz2012tecnicas}
\parencite{Taylor1958NoiseCI}
\parencite{Tsiou2008Noise}
\parencite{text-to-speech-ibm}
\parencite{West2008NoiseIH}}

%-----------------Bibliografia-----------------
\newpage

\printbibliography[title={Bibliografia}]

%-----------------ANEXO-----------------
\newpage
\appendix
\section*{ANEXO I. \quad FORMATO INTERNO} 
    \input{secciones/9.anexo}
\end{document}


